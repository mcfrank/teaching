\documentclass[12pt]{article}
\usepackage{amsmath}
\title{Details of the derivation of information gain for multiple samples, ``Stochastic Modeling of the Dynamics of Multi-Classroom Education Using Teaching Games''}
\date{}

\begin{document}
\maketitle

As described in the text, we are interested in computing

\begin{equation}
\label{eq:ig}
IG(e) = D_{KL} ( B_T || B_{S} )  - D_{KL} (B_T ||B_{S+e} ) 
\end{equation}

\noindent where the divergence measure is computed in closed form for e.g., $B_T$ and $B_S$, as

\begin{equation}
\label{eq:dkl}
\begin{split}
D_{KL} (B_T ||B_{S} )  = & \log( \frac{B(\alpha_{S},\beta_{S})}{B(\alpha_{T},\beta_{T})}) +  (\alpha_T - \alpha_S) \psi (\alpha_T) \\ 
& + (\beta_T - \beta_S) \psi (\beta_T) +  (\alpha_T - \alpha_S + \beta_T - \beta_S) \psi (\alpha_T + \beta_T). \\
\end{split}
\end{equation}

\noindent where $\psi$ denotes the digamma function and $B(a,b)$ denotes the beta function. We can substitute Equation \ref{eq:dkl} into Equation \ref{eq:ig} to get

\begin{equation}
\begin{split}
IG(e)  = & \log( \frac{B(\alpha_{S},\beta_{S})}{B(\alpha_{T},\beta_{T})}) + (\alpha_T - \alpha_S) \psi (\alpha_T)  \\ 
& + (\beta_T - \beta_S) \psi (\beta_T) +  (\alpha_T - \alpha_S + \beta_T - \beta_S) \psi (\alpha_T + \beta_T)  \\
& - \log( \frac{B(\alpha_{S+e},\beta_{S+e})}{B(\alpha_{T},\beta_{T})}) - (\alpha_T - \alpha_{S+e}) \psi (\alpha_T) \\ 
& - (\beta_T - \beta_{S+e}) \psi (\beta_T) - (\alpha_T - \alpha_{S+e} + \beta_T - \beta_{S+e}) \psi (\alpha_T + \beta_T)
\end{split}
\label{eq:step2}
\end{equation}

Consider the case where $e$ is a series of $h$ 1's (heads) and $t$ 0's (tails). Then $\alpha_{S+e}= \alpha_S + h$ and $\beta_{S+e}= \beta_{S} + t$, so we can simplify Equation \ref{eq:step2} to

\begin{equation}
\begin{split}
IG(e)  = & \log( \frac{B(\alpha_{S},\beta_{S})}{B(\alpha_{T},\beta_{T})}) - \log( \frac{B(\alpha_{S}+h,\beta_{S}+t)}{B(\alpha_{T},\beta_{T})}) \\ 
& + (\alpha_T - \alpha_S) \psi (\alpha_T) +  (\beta_T - \beta_S) \psi (\beta_T) \\
& + (\alpha_T - \alpha_S + \beta_T - \beta_S) \psi (\alpha_T + \beta_T)  \\
& - (\alpha_T - \alpha_{S} - h) \psi (\alpha_T) - (\beta_T - \beta_S - t) \psi (\beta_T) \\
& - (\alpha_T - \alpha_{S} - h + \beta_T - \beta_{S} - t) \psi (\alpha_T + \beta_T) \\
= & \log( \frac{B(\alpha_{S},\beta_{S})}{B(\alpha_{T},\beta_{T})} \cdot \frac{B(\alpha_{T},\beta_{T})}{B(\alpha_{S}+h,\beta_{S}+t)}) \\ 
& + (h) \psi (\alpha_T) + (t) \psi (\beta_T) + (h + t) \psi (\alpha_T + \beta_T). \\
= & \log( \frac{B(\alpha_{S},\beta_{S})}{B(\alpha_{S}+h,\beta_{S}+t)}) \\ 
& + (h) \psi (\alpha_T) + (t) \psi (\beta_T) + (h + t) \psi (\alpha_T + \beta_T). \\
\end{split}
\end{equation}

\noindent And, since 

\begin{equation}
B(a,b) = \frac{\Gamma(a)\Gamma(b)}{\Gamma(a+b)},
\end{equation}

\noindent we can rewrite the first term and reduce:

\begin{equation}
\begin{split}
IG(e) = & \log( \frac{\frac{\Gamma(\alpha_S)\Gamma(\beta_S)}{\Gamma(\alpha_S+\beta_S)}}{\frac{\Gamma(\alpha_S + h)\Gamma(\beta_S + t)}{\Gamma(\alpha_S+\beta_S + h + t)}})   \\
& + (h) \psi (\alpha_T) + (t) \psi (\beta_T) + (h + t) \psi (\alpha_T + \beta_T). \\
=& \log( \frac{\Gamma(\alpha_S)\Gamma(\beta_S) \Gamma(\alpha_S+\beta_S + h + t)}{ \Gamma(\alpha_S + h)\Gamma(\beta_S + t) \Gamma(\alpha_S+\beta_S)})   \\
& + (h) \psi (\alpha_T) + (t) \psi (\beta_T) + (h + t) \psi (\alpha_T + \beta_T). \\
=& \log( \frac{\Gamma(\alpha_S)}{\Gamma(\alpha_S + h)}) + \log( \frac{\Gamma(\beta_S)}{\Gamma(\beta_S + t)}) + \log(\frac{\Gamma(\alpha_S+\beta_S + h + t)}{  \Gamma(\alpha_S+\beta_S)})   \\
& + (h) \psi (\alpha_T) + (t) \psi (\beta_T) + (h + t) \psi (\alpha_T + \beta_T). \\
=& -\log( \frac{\Gamma(\alpha_S + h)}{\Gamma(\alpha_S)}) - \log( \frac{\Gamma(\beta_S + t)}{\Gamma(\beta_S)}) + \log(\frac{\Gamma(\alpha_S+\beta_S + h + t)}{  \Gamma(\alpha_S+\beta_S)})   \\
& + (h) \psi (\alpha_T) + (t) \psi (\beta_T) + (h + t) \psi (\alpha_T + \beta_T). \\
\end{split}
\end{equation}

\noindent When $h=0$, the first log term reduces to 0. When $t=0$, the second log term reduces to 0. Otherwise, since for positive integer $n$, 

\begin{equation}
\frac{\Gamma(x + n)}{\Gamma(x)} = \prod^{n-1}_{k=0}{(x+k)},
\end{equation}

\noindent we can reduce the previous formulation a bit further when $h \neq 0$ and $t \neq 0$, to

\begin{equation}
\begin{split}
IG(e) = &- \log(\prod^{h-1}_{i=0}(\alpha_S + i)) - \log(\prod^{t-1}_{j=0}(\beta_S + j)) + \log(\prod^{h + t -1}_{k=0}(\alpha_S + \beta_S + k)) \\
& + (h) \psi (\alpha_T) + (t) \psi (\beta_T) + (h + t) \psi (\alpha_T + \beta_T). \\
\end{split}
\end{equation}

\noindent For computation, to prevent overflow we calculate the series equivalently as follows:

\begin{equation}
\begin{split}
IG(e) = &\sum^{h + t -1}_{k=0}\log(\alpha_S + \beta_S + k) - \sum^{h-1}_{i=0}\log(\alpha_S + i) - \sum^{t-1}_{j=0}\log(\beta_S + j) \\
& + (h) \psi (\alpha_T) + (t) \psi (\beta_T) + (h + t) \psi (\alpha_T + \beta_T). \\
\end{split}
\end{equation}

\end{document}